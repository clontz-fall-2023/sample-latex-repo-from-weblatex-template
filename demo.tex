% This is a simple sample document.  For more complicated documents take a look in the exercise tab. Note that everything that comes after a % symbol is treated as comment and ignored when the code is compiled.

\documentclass{article} % \documentclass{} is the first command in any LaTeX code.  It is used to define what kind of document you are creating such as an article or a book, and begins the document preamble

\usepackage{amsmath} % \usepackage is a command that allows you to add functionality to your LaTeX code

\usepackage{hyperref}

\title{Homework Set MA 237} % Sets article title
\author{Steven Clontz} % Sets authors name
\date{\today} % Sets date for date compiled

% The preamble ends with the command \begin{document}
\begin{document} % All begin commands must be paired with an end command somewhere
\maketitle % creates title using information in preamble (title, author, date)

\section{LE2 v1} % creates a section

\href{https://teambasedinquirylearning.github.io/linear-algebra/2023/exercises/#/bank/LE2/1/}{LE2/1}

%%%%% SpaTeXt Commands %%%%%
\providecommand{\stxKnowl}{}\renewcommand{\stxKnowl}[1]{#1}
\providecommand{\stxOuttro}{}\renewcommand{\stxOuttro}[1]{#1}
\providecommand{\stxTitle}{}\renewcommand{\stxTitle}[1]{#1}
% Comment next line to show outtros
\renewcommand{\stxOuttro}[1]{}
%%%%%%%%%%%%%%%%%%%%%%%%%%%%
\stxKnowl{
  \begin{enumerate}
    \item
          \stxKnowl{
            For each of the following matrices, explain why it is not in reduced row echelon form.

            \begin{enumerate}
              \item
                    \stxKnowl{
                      \[A = \left[\begin{array}{ccccc} 0 & 0 & 1 & 0 & -2 \\ 1 & 5 & 0 & -2 & 1 \\ 0 & 0 & 0 & 0 & 0 \end{array}\right]\]

                      \stxOuttro{
                        \(A=\left[\begin{array}{ccccc} 0 & 0 & 1 & 0 & -2 \\ 1 & 5 & 0 & -2 & 1 \\ 0 & 0 & 0 & 0 & 0 \end{array}\right]\) is not in reduced row echelon form because   the pivots are not descending to the right.

                      }
                    }

              \item
                    \stxKnowl{
                      \[B = \left[\begin{array}{ccccc} 1 & -6 & 3 & 0 & -1 \\ 0 & 0 & 0 & 7 & 14 \\ 0 & 0 & 0 & 0 & 0 \end{array}\right]\]

                      \stxOuttro{
                        \(B=\left[\begin{array}{ccccc} 1 & -6 & 3 & 0 & -1 \\ 0 & 0 & 0 & 7 & 14 \\ 0 & 0 & 0 & 0 & 0 \end{array}\right]\) is not in reduced row echelon form because    the pivots are not all \(1\).

                      }
                    }

              \item
                    \stxKnowl{
                      \[C = \left[\begin{array}{ccccc} 1 & 7 & -4 & 1 & 12 \\ 0 & 1 & -1 & 0 & 2 \\ 0 & 0 & 0 & 0 & 0 \end{array}\right]\]

                      \stxOuttro{
                        \(C=\left[\begin{array}{ccccc} 1 & 7 & -4 & 1 & 12 \\ 0 & 1 & -1 & 0 & 2 \\ 0 & 0 & 0 & 0 & 0 \end{array}\right]\) is not in reduced row echelon form because  not every entry above and below each pivot is zero.

                      }
                    }

            \end{enumerate}
          }

    \item
          \stxKnowl{
            Show step by step why \[\operatorname{RREF}\left[\begin{array}{ccccc} 4 & 4 & 3 & 18 & -11 \\ -3 & -3 & 1 & -7 & 5 \\ 3 & 3 & 3 & 15 & -9 \end{array}\right] = \left[\begin{array}{ccccc} 1 & 1 & 0 & 3 & -2 \\ 0 & 0 & 1 & 2 & -1 \\ 0 & 0 & 0 & 0 & 0 \end{array}\right]\]

            \stxOuttro{
              \[\left[\begin{array}{ccccc} 4 & 4 & 3 & 18 & -11 \\ -3 & -3 & 1 & -7 & 5 \\ 3 & 3 & 3 & 15 & -9 \end{array}\right] \sim \cdots \sim \left[\begin{array}{ccccc} 1 & 1 & 0 & 3 & -2 \\ 0 & 0 & 1 & 2 & -1 \\ 0 & 0 & 0 & 0 & 0 \end{array}\right]\]

            }
          }

  \end{enumerate}
}

\end{document} % This is the end of the document